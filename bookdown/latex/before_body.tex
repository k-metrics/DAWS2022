% --- 空ページにヘッダーを設定する(機能していない) --------------------------
\makeatletter
\def\emptypage@emptypage{
    \hbox{}
    \thispagestyle{headings}
    \newpage
}
\def\cleardoublepage{
        \clearpage
        \if@twoside
            \ifodd\c@page
                % do nothing
            \else
                \emptypage@emptypage
            \fi
        \fi
    }
\makeatother

% --- 参考文献一覧を目次に表示させる ------------------------------------------
\makeatletter
\renewenvironment{thebibliography}[1]
{\chapter*{\bibname\@mkboth{\bibname}{\bibname}}
   \addcontentsline{toc}{chapter}{\bibname}%
   \list{\@biblabel{\@arabic\c@enumiv}}%
        {\settowidth\labelwidth{\@biblabel{#1}}%
         \leftmargin\labelwidth
         \advance\leftmargin\labelsep
         \@openbib@code
         \usecounter{enumiv}%
         \let\p@enumiv\@empty
         \renewcommand\theenumiv{\@arabic\c@enumiv}}%
   \sloppy
   \clubpenalty4000
   \@clubpenalty\clubpenalty
   \widowpenalty4000%
   \sfcode`\.\@m}
  {\def\@noitemerr
    {\@latex@warning{Empty `thebibliography' environment}}%
   \endlist}
\makeatother

% --- 別名索引の定義 ----------------------------------------------------------
% https://texwiki.texjp.org/?%E7%B4%A2%E5%BC%95%E4%BD%9C%E6%88%90#l86bc3d9
\renewcommand{\seename}{$\rightarrow$}
% 別名定義
% \index{A|see{B}} と指定すれば索引にて A→B と表記される
\index{Colab|see{Google Colaboratory}}
\index{Google Colab|see{Google Colaboratory}}
\index{iris|see{iris dataset}}
\index{Rcmdr|see{R Commander}}
% - あ 
% - か 
% - さ 
% - た 
% - な 
% - は 
\index{ぶんさん@分散|see{不偏分散}}
\index{へいきん@平均|see{算術平均}}
% - ま 
\index{モード|see{最頻値}}
% - や 
% - ら 
% - わ 
